\documentclass{article}
\usepackage[utf8]{inputenc}
\usepackage{booktabs}
\usepackage{float}
\usepackage[a4paper, width=150mm, top=25mm, bottom=25mm, bindingoffset=6mm]{geometry} 
\usepackage{verbatim}


\title{BDPP Lab 3 \\ Kubernetes}
\author{Karl-Johan Djervbrant}

% Hide section numbering
\makeatletter
\def\@seccntformat#1{%
  \expandafter\ifx\csname c@#1\endcsname\c@section\else
  \csname the#1\endcsname\quad
  \fi}
\makeatother

\begin{document}

    \maketitle

    \section{Answers to Task 1}
    \begin{enumerate}
        \item a) Describe differences between master node and worker nodes? \\
        The master coordinates the cluster and the nodes are the workers which run the applications.
        \\\\
        b) What version of minikube is used in the tutorial? \\
        Minikube version: v1.8.1
        \item a) Describe kubectl? \\
        Kubectl is the commandline interface used to control and manage the cluster.
        \\\\
        b) What is the output of \texttt{kubectl get nodes}?\\\\
        \begin{tabular}{lllll}
            NAME & STATUS & ROLES & AGE & VERSION \\
            minikube & Ready & master & 12m & v1.17.3
        \end{tabular}
        \\\\
        c) What is the ouput of \texttt{kubectl get deployments}\\\\
        \begin{tabular}{lllll}
            NAME & STATUS & UP-TO-DATE & AVAILABLE & AGE \\
            kubernetes-bootcamp & 1/1 & 1 & 1 & 61s
        \end{tabular}
        \item a) What is a pod, describe its entities? \\
        Pods run inside the Kubernetes application and is the basic execution unit and can be controlled via 
        the API. It's like Docker container and can be services such as shared storage and networking.
        \\\\
        b) What information do the \texttt{kubectl describe pods} command give? \\
        It returns information such as, name, starttime, ip (server address), the running image etc.
        \item a) Describe what a service is in Kubernetes? \\
        A service is a abstraction layer which enables the user to expose a node ip to the cluster. 
        \\\\
        b) What do the \texttt{kubectl expose deployment/kubernetes-bootcamp --type="NodePort" --port 8080} command do? \\
        It uses NAT to expose a service on a set port, it enables a service to be accessed from outside
        the cluster. In this case it exposes the kubernetes-bootcamp on port \texttt{8080}.
        \item a) How do labels and Label Selector objects relate to a Service? \\
        You can set a label on your pod and interact with it via the label instead. You can delete a service with the 
        \texttt{-l} flag and then enter the label to delete that service.
        \\\\
        b) What is the output \texttt{kubectl get rs} command?\\\\
        \begin{tabular}{lllll}
            NAME & DESIRED & CURRENT & READY & AGE \\
            kubernetes-bootcamp-765bf4c7b4 & 1 & 1 & 1 & 2m5s
        \end{tabular}
        \\\\
        This is the ReplicaSet.
        \item a) Describe rolling updates in Kubernetes? \\
        It enables the user to update the deployments (pods) without downtime.
        \item b) Whats does the \texttt{kubectl set image deployments/kubernetes-bootcamp kubernetes-bootcamp=jocatalin/kubernetes-bootcamp:v2} command do? \\
        It updates the image of the pods to version 2.
    \end{enumerate}
\end{document}